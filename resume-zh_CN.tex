% !TEX TS-program = xelatex
% !TEX encoding = UTF-8 Unicode
% !Mode:: "TeX:UTF-8"

\documentclass{resume}
\usepackage{zh_CN-Adobefonts_external} % Simplified Chinese Support using external fonts (./fonts/zh_CN-Adobe/)
%\usepackage{zh_CN-Adobefonts_internal} % Simplified Chinese Support using system fonts
\usepackage{linespacing_fix} % disable extra space before next section
\usepackage{cite}

\begin{document}
\pagenumbering{gobble} % suppress displaying page number

\name{文佳源}

% {E-mail}{mobilephone}{homepage}
% be careful of _ in emaill address
\contactInfo{(+86) 136-6810-0193}{648137125@qq.com}{嵌入式软件开发}{GitHub @peach-1s-me}
% {E-mail}{mobilephone}
% keep the last empty braces!
%\contactInfo{xxx@yuanbin.me}{(+86) 131-221-87xxx}{}
 
\section{个人总结}
本人在校成绩优秀、乐观积极,对待工作认真负责、自我驱动力强、热爱尝试新事物。有牢固的专业基础知识和良好的编码习惯,善于团队合作,了解软件工程的基本理论和实践方法。在校期间长期从事嵌入式系统及其应用的相关研究,对嵌入式技术和其工程化解决方案有浓厚兴趣。\textbf{现就读于电子科技大学。}

% \section{\faGraduationCap\ 教育背景}
\section{教育背景}
\datedsubsection{\textbf{电子科技大学},软件工程,\textit{在读硕士研究生}}{2023.9 - 2026.6}
\ \textbf{嵌入式智能计算实验室},研究方向\textbf{实时操作系统},GPA 3.8/4.0,课程:\textbf{矩阵理论、图论、高级软件工程}等
\datedsubsection{\textbf{电子科技大学},软件工程,\textit{工学学士}}{2018.9 - 2022.6}
\ 专业方向\textbf{嵌入式系统},CET-6,GPA 3.7/4.0,课程:\textbf{嵌入式操作系统、arm体系结构、微机接口技术}等

% \section{\faCogs\ IT 技能}
\section{技术能力}
% increase linespacing [parsep=0.5ex]
\begin{itemize}[parsep=0.2ex]
  \item \textbf{编程语言和工具}: C/asm, keil/eclipse/GCC/openocd/shell/Git/Makefile/CMake
\end{itemize}

% \end{itemize}

\section{实习经历}
\datedsubsection{\textbf{深圳大疆创新有限公司 | 车载部门}, 嵌入式实习生}{2021.1-2021.7}
\begin{itemize}
  \item 参与开发调试工具的维护,优化Makefile脚本和RTOS的shell功能,设计实现CPU使用率统计
  \item 负责部分单元测试,针对车规需求完成部分模块的单元测试(白盒/黑盒)用例设计和执行测试
  \item 负责ADC驱动开发,实现MCU端电压数据采集,完成裸机寄存器到RTOS驱动框架的开发适配
\end{itemize}

% \begin{onehalfspacing}
% \end{onehalfspacing}

% \datedsubsection{\textbf{DID-ACTE} 荷兰莱顿}{2015年}
% \role{本科毕业设计}{LIACS 交换生}
% 利用结巴分词对中国古文进行分词与词性标注,用已有领域知识训练形成 classifier 并对结果进行调优
% \begin{onehalfspacing}
% \begin{itemize}
%   \item 利用结巴分词对中国古文进行分词与词性标注
%   \item 利用已有领域知识训练形成 classifier, 并用分词结果进行测试反馈
%   \item 尝试不同规则,对 classifier 进行调优
% \end{itemize}
% \end{onehalfspacing}

\section{参与项目}
% increase linespacing [parsep=0.5ex]
% \datedsubsection{\textbf{项目名称},\small{独立开发/参与开发/文档编写}}{2021.3-   \textbf{至今}}
% 介绍,链接
% \begin{itemize}
%     \item aaa
% \end{itemize}
\datedsubsection{\textbf{珊瑚aCoral操作系统},{\normalsize 参与开发}}{2023.6 -\enspace\textbf{至今}}
\datedline{电子科技大学嵌入式智能实验室开发的实时操作系统}{https://github.com/EIC-UESTC/aCoral}
\begin{itemize}
    \item 作为项目负责人,负责开发和维护多核主版本操作系统,制定代码规范和git提交流程等
    \item 实现多核EDF调度算法、基于Cortex-A9平台开发MMU内存映射、LWIP适配以及完成代码重构
\end{itemize}

\datedsubsection{\textbf{CatOS嵌入式操作系统},{\normalsize 独立开发}}{2021.3 -\enspace\textbf{至今}}
\datedline{一个简单的嵌入式实时操作系统}{https://github.com/peach-1s-me/CatOS}
\begin{itemize}
    \item 基于优先级调度,任务间支持信号量、互斥量以及消息队列,实现优先级继承解决优先级反转问题
    \item 内核提供可移植接口,包含简单的设备驱动框架、动态内存管理和shell,提供多等级日志输出功能
    \item 支持位图结构、差分链表、环形缓冲区等数据结构和操作,提高效率,便于内核扩展和应用开发
\end{itemize}

% \datedsubsection{\textbf{基于 CatOS 的平衡小车},{\normalsize 独立开发}}{2025.4 - 2025.4}
% \datedline{基于 CatOS 操作系统的平衡小车}{https://github.com/peach-1s-me/BalanceCar}
% \begin{itemize}
%     \item 将系统功能划分为传感器采集、电机控制和显示三个任务,利用信号量等机制完成多任务的协同
% \end{itemize}

\datedsubsection{\textbf{自动贩卖咖啡机},{\normalsize 参与开发}}{2024.3 - 2024.4}
\datedline{小程序下单自动贩卖咖啡机}{}
\begin{itemize}
    \item 负责开发嵌入式通信部分,搭建和接入阿里云mqtt服务器,利用cJSON封装和解析json数据
    \item 系统实现小程序下单后订单数据通过服务器发送到咖啡机,解析咖啡类型、胶囊坐标等信息
\end{itemize}

\datedsubsection{\textbf{文件传输软件},{\normalsize 独立开发}}{2020.11 - 2020.12}
\datedline{linux下的加密文件传输软件}{https://github.com/peach-1s-me/opensslTransmit}
\begin{itemize}
    \item 实现文件收发客户端,使用openSSL库的完成TCP传输过程的加密,并使用QT实现GUI界面
\end{itemize}

\datedsubsection{\textbf{机房监控系统},{\normalsize 参与开发}}{2019.11 - 2019.12}
\datedline{基于树莓派实现一个机房监控系统}{}
\begin{itemize}
    \item 作为组长,负责嵌入式端摄像头图像数据采集、推流以及基于nginx搭建rtmp服务器
    \item 系统实现嵌入式端采集图像和传感器数据,客户端从服务器获取数据并展示,支持异常报警
    
\end{itemize}

\section{竞赛获奖}
% increase linespacing [parsep=0.5ex]
\begin{itemize}[parsep=0.2ex]
%   \item LeetCodeOJ Solutions, \textit{https://github.com/hijiangtao/LeetCodeOJ}
  \item 第十九届“挑战杯”全国大学生课外学术科技作品竞赛\textbf{全国特等奖},2024年11月
\end{itemize}

%% Reference
%\newpage
%\bibliographystyle{IEEETran}
%\bibliography{mycite}
\end{document}
